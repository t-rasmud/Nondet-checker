\htmlhr
\chapter{NonDeterminism Checker\label{nondeterminism-checker}}
The non-determinism checker identifies locations in source code that might result in different execution outcomes. 
In particular, it checks for non-determinism that pertains to the use of JAVA Collections.
Consider the following example:
\begin{Verbatim}
	Set<Integer> st = new HashSet<Integer>(); 
	for(int i = 0 ; i < 10 ; i++)
		st.add(i);
	Iterator<Integer> it = st.iterator();
	while(iterator.hasNext())
		System.out.println(iterator.next());

\end{Verbatim}

The code above is non-deterministic because a HashSet is not ordered and its elements could be printed in different orders across executions.
On the other hand, certain Set functions like 'contains' always return deterministic results.

To run the NonDeterminism Checker, supply the
\code{-processor NonDeterminismChecker}
command-line option to javac.
For example: ./bin/javac -cp /Users/rashmi/jsr308/checker-framework/checker/build -processor org.checkerframework.checker.nondeterminism.NonDeterminismChecker /Users/rashmi/NDCheckerTest/src/TestListUnsafe2.java

\section{NonDeterminism annotations for collections\label{nondeterminism-annotations-collections}}

The NonDeterminism type system uses the following annotations:
\begin{itemize}
\item
  \refqualclass{checker/nondeterminism/qual}{ValueNonDet} indicates
  that the collection can have possibly different values in two different executions.
\item
  \refqualclass{checker/nondeterminism/qual}{OrderNonDet} indicates that
  the collection will have the same elements, but in a possibly different order in two different executions.
  \code{@ValueNonDet} is a supertype of \code{@OrderNonDet}.
 \item
  \refqualclass{checker/nondeterminism/qual}{Det} indicates that
  the collection will have the same elements, in the same order across all executions.
  \code{@OrderNonDet} is a supertype of \code{@Det}.
   It is the default qualifier.
\item
  \refqualclass{checker/nondeterminism/qual}{PolyDet and PolyDet2} are qualifiers that are
  polymorphic over nondeterminism (see Section~\ref{qualifier-polymorphism}).
\end{itemize}

\section{NonDeterminism annotations for all other types\label{nondeterminism-annotations-elements}}

\begin{itemize}
\item
  \refqualclass{checker/nondeterminism/qual}{ValueNonDet} indicates
  that this type can possibly have different values across different executions. 
 \item
  \refqualclass{checker/nondeterminism/qual}{Det} indicates that
  the value remains the same in any execution.
\end{itemize}

Note that non-collections may not be annotated as \code{OrderNonDet}

% TODO: one could add a String[] value to @Untainted to distinguish different
% values, eg @Untainted{``SQL''} versus @Untainted{``HTML''}.

% LocalWords:  quals untaint PolyTainted mis untainting sanitization java
%%  LocalWords:  TaintingChecker untrusted unvalidated
